%%% Поля и разметка страницы %%%
\documentclass[a4paper,12pt]{article}
\usepackage{lscape}		% Для включения альбомных страниц

%%% Кодировки и шрифты %%%
\usepackage{cmap}						% Улучшенный поиск русских слов в полученном pdf-файле
\usepackage[T2A]{fontenc}				% Поддержка русских букв
\usepackage[utf8]{inputenc}				% Кодировка utf8
\usepackage[english, russian]{babel}	% Языки: русский, английский
%\usepackage{pscyr}						% Красивые русские шрифты

%%% Математические пакеты %%%
\usepackage{amsthm,amsfonts,amsmath,amssymb,amscd} % Математические дополнения от AMS

%%% Оформление абзацев %%%
\usepackage{indentfirst} % Красная строка
\usepackage{alltt}

%%% Цвета %%%
\usepackage[usenames]{color}
\usepackage{color}
\usepackage{colortbl}

%%% Таблицы %%%
\usepackage{longtable}					% Длинные таблицы
\usepackage{multirow,makecell,array}	% Улучшенное форматирование таблиц

%%% Общее форматирование
\usepackage[singlelinecheck=off,center]{caption}	% Многострочные подписи
\usepackage{soul}									% Поддержка переносоустойчивых подчёркиваний и зачёркиваний

%%% Библиография %%%
\usepackage{cite} % Красивые ссылки на литературу

%%% Гиперссылки %%%
\usepackage[plainpages=false,pdfpagelabels=false]{hyperref}
\definecolor{linkcolor}{rgb}{0.9,0,0}
\definecolor{citecolor}{rgb}{0,0.6,0}
\definecolor{urlcolor}{rgb}{0,0,1}
\hypersetup{
    colorlinks, linkcolor={linkcolor},
    citecolor={citecolor}, urlcolor={urlcolor}
}

%%% Изображения %%%
\usepackage{graphicx}		% Подключаем пакет работы с графикой
\graphicspath{{images/}}	% Пути к изображениям

%%% Выравнивание и переносы %%%
\sloppy					% Избавляемся от переполнений
\clubpenalty=10000		% Запрещаем разрыв страницы после первой строки абзаца
\widowpenalty=10000		% Запрещаем разрыв страницы после последней строки абзаца

%%% Библиография %%%
\makeatletter
\bibliographystyle{utf8gost705u}	% Оформляем библиографию в соответствии с ГОСТ 7.0.5
\renewcommand{\@biblabel}[1]{#1.}	% Заменяем библиографию с квадратных скобок на точку:
\makeatother

%%% Колонтитулы %%%
\let\Sectionmark\sectionmark
\def\sectionmark#1{\def\Sectionname{#1}\Sectionmark{#1}}
\makeatletter
\newcommand*{\currentname}{\@currentlabelname}
\renewcommand{\@oddhead}{\it \vbox{\hbox to \textwidth%
    {\hfil Белокрыс-Федотов А.И. --- Форматы данных для записи сеток\hfil\strut}\hbox to \textwidth%
    {\today \hfil \thesection~\Sectionname\strut}\hrule}}
\makeatother

%%%%%%%%%%%%%%%%%%%%%%%%%%%%%%%%%%%%%%%%%%%%%%%%%%%%%%%%%%%%%%%%%%%%%%%%%%%%%%%%%%%
\begin{document}

{~}\bigskip
\begin{center}
\Huge{Форматы данных для записи сеток}
\end{center}

%\tableofcontents 

\section{Введение}
Современные инструменты решения задач вычислительной математики немыслимы без чтения из файлов начальных условий и записи результатов вычислений. На текущий момент существует огромное кол-во раличных форматов для записи тех или иных расчётных данных: структуированные и не структуированные, однородные и не однородные сетки, сетки с данными в точках или в ячейках, многоблочные и иерархические сетки, с данными в виде значений, цветов, нормалей, потоков и т.д.. Под разные задачи подходят разные виды форматов.

Данная работы пытается сделать, по возможности, полный обзор тех форматов, с которыми мне довелось столкнуться. К сожалению, некоторые форматы проприетарны и работы с ними возможна только через специализированный софт, для некоторых требуется подключение сложной математики и геометрии (сплайновые и иерархические), такие форматы будут указаны только в скользь.

Для начала стоит ввести общие понятия и терминалогию. Как правило задачи решаются в двумерном или трехмерном пространстве, но большинство форматов хранят трехмерные координаты, поэтому, если не указано иное, будем считать что мы работаем в трехмерном пространстве.

\begin{itemize}
\item Точка --- точка в $R^3$.
\item Ячейка --- точка, ребро (отрезок), многоугольник, многогранник.
\item Грань ячейки --- для многоугольника - ребро/отрезок, для многогранника - грань/многоугольник, для ребра - конец/точка, для точки не определено.
\item Сетка --- набор ячеек, точки сетки - вершины ячеек.
\end{itemize}

Как правило в коде сетка задается следующими данными:
\begin{itemize}
\item кол-во точек в сетке NP;
\item массив точек P, массив размерности [NP][3] или [NP*3];
\item кол-во ячеек;
\item ячейки, каждая ячейка задается типом, кол-вом вершин в ячейке и массивом индексов вершин в массиве точек.
\end{itemize}

Важно отметить, что некоторые типы ячеек могут иметь различную нумирацию вершин, поэтому нужно заранее договориться о "правильной" нумирации. Например для гексаэдров распространены два типа нумирации.

Картинка с блоком и гексом.

Так же стоит обратить внимание на то, что индексы в массивах точек могут начинаться с 0 и с 1. Далее в работе считается что индексы нумируются с 0 до NP-1.

Для задания начальных условий в задачах аэрогидродинамики требуется определения различных граничных условий на разных частях общей границы тела, для этого к данным добавляются списки граней ячеек разбитые на группы, например, грани входящего/уходящего потока, грани обтекаемого тела. Наиболее удобный способ записи граничных условий таков:
\begin{itemize}
\item кол-во различных граничных условий NB
\item для каждого граничного условия:
\begin{itemize}
\item кол-во граней для данного граничного условия
\item список граней для данного граничного условия в одном из представлений
\begin{itemize}
\item грань как ячейка, те для тетраэдральной сетки - набор треугольников
\item пара индексов --- номер ячейки и локальный номер грани ячейки
\end{itemize}
\end{itemize}
\end{itemize}

Последний вариант предпочтительней, так как в данной структуре явно указывается связь между ячейками сетки и граничными условиями, в то время как в первом --- данную связь необходимо восстанавливать.

Для вывода результатов решения необходимо запись данных расчетов вместе с сеткой. Для начала определим с какими данными мы будем работать.
\begin{itemize}
\item материал --- целочисленное значение, как правило не отрицательное. Может под собой нести информацию об индексе, маске, цвете в палитре цветов и тд.
\item фунция ---- вещественное значение. Обычно используется для записи результатов расчета безразменых величин: плотность, давление, температура; или характеристик сетки: качество, объем,
\item текстурные координаты --- два вещественных числа от 0.0 до 1.0. Используются для привязки текстуры к поверхности. Как правио в научных задачах не используется.
\item цвет --- три или четыре целых числа от 0 до 255 или вещественных числа от 0.0 до 1.0. Используется для задания цвета. Первые три числа задают координаты в RGB пространстве цветов, опциаональное четвертое число прозрачность.
\item вектор --- три вещественных числа. Используется для хранения таких величин как силы, скорости, потоки и т.д..
\end{itemize}

Данные можно привязать к точкам, ячейкам и граничным условиям, причем в любом кол-ве и наборе. Однако в большинстве форматах можно записать только один набор данных. Это нужно учитывать.

\clearpage
\section{Формат OFF}

Формат OFF (Object File Format) - ASCII формат для представления полигональной поверхности. Есть бинарный вариант.

В этом формате можно хранить только многоугольные ячейки и ограниченный набор данных, но из-за простоты формата им бывает удобно пользоваться. Нумирация вершин начинается с 0.

Wolfram (http://reference.wolfram.com/mathematica/ref/format/OFF.html):
\begin{quote}
OFF 3D geometry format.
Used for storing and exchanging 3D models.
OFF is an acronym for Object File Format.
Occasionally called COFF if color information is present.
Related to NOFF and CNOFF.
ASCII or binary format.
Represents a single 2D or 3D object.
Stores a collection of planar polygons with possibly shared vertices.
Supports polygon and vertex colors and opacity specifications.
\end{quote}

Geomview (http://www.geomview.org/docs/html/OFF.html):
\begin{quote}
OFF files (name for "object file format") represent collections of planar polygons with possibly shared vertices, a convenient way to describe polyhedra. The polygons may be concave but there's no provision for polygons containing holes. 
\end{quote}

Общий синтаксис формата такой:

\begin{alltt}
[ST][C][N][4][n]OFF	# Заголовок
[Ndim]		# Размерность пространства
NVertices  NFaces  NEdges
x[0]  y[0]  z[0]	# Вершины
...
x[NVertices-1]  y[NVertices-1]  z[NVertices-1]
Nv[0]  v[0][0] v[0][1] ... v[0][Nv-1]  color[0] # Грани
...
Nv[NFaces-1]  v[NFaces-1][0] ... v[NFaces-1][Nv-1]  color[NFaces-1]
\end{alltt}

А теперь по частям. Заголовок с модификаторами:
\begin{alltt}
[ST][C][N][4][n]OFF
\end{alltt}
Каждый указанный модификатор к заголовку указывает, какие данные задаются в точках:
\begin{itemize}
\item ST - текстурные коодинаты
\item C - цвет
\item N - вектор (нормаль)
\end{itemize}

Модификатор 4 указыват, что вершины имеют 4 компоненты, 4ая компонента может трактоваться как 4ое измерение или как функция заданная в вершине. Модификатор n указыват, что размерность пространства будет указано отдельно как Ndim. Стоит отметить, что объединение модификаторов 4n задает для каждой точки Ndim+1 компонент. Если указаны модификаторы Nn, размерность векторов в точках равна Ndim, модификатор 4 не влияет на это.

\begin{alltt}
[Ndim]
\end{alltt}
Опциональный параметр, который используется только с модификтором n, во всех других случиях или отстутсвует, или игнорируется.

\begin{alltt}
NVertices  NFaces  NEdges
\end{alltt}
Кол-во вершин, граней и рёбер. На данный момент почти все программы игнорируют третий параметр, и обычно его ставят 0.

Далее для каждой вершины идут её координаты и данные связанные с ней.
\begin{alltt}
x y z [nx ny nz] [( c | r g b [a] )] [ t u ]
\end{alltt}
Где:
\begin{itemize}
\item x, y, z --- компоненты вершины, при использовании модификаторов 4 или n их может быть иное кол-во.
\item nx, ny, nz ---  компоненты вектора, при модификаторе n их будет Ndim
\item c --- целочисленный индекс цвета, который следует трактовать как материал
\item r, g, b, a --- цвет заданный в вершине
\item t, u --- текстурные координаты
\end{itemize}

Важно отметить, что порядок данных фиксирован. Это позволяет задвать цвет разными способами. Например задан заголовок STC4OFF, тогда если в строчке 7 чисел, то первые 4 это компоненты точки, последние 2 --- текстурные координаты, а оставшиеся одно число - индекс цвета.

Далее для каждой ячейки записываются её размер, список вершин и цвет
\begin{alltt}
Nv v[0] v[1] ... v[Nv] [( c | r g b [a] )]
\end{alltt}

Так как в формате OFF ячейки являются многогранниками, тип ячейки напрямую связан с кол-во вершин ячейки: для трегульника $Nv==3$, для четырехугольника $Nv==4$.

Trick. Теоритически можно задавать $Nv==1$ для ячеек типа точек, и $Nv==2$ для отрезков, но такая возможно оффициально не документированна, и многие программы будут пропускать такие ячейки.

Если цвет у ячейки не указан, то ячейка помечается цветом по умолчанию из палитры цветов.
Многие программы, например Geomview, целочисленный цвет будут преобразовывать в нормальный используя палитру цветов, для Geomview она задается в файле `cmap.fmap' в директории '%GeomviewPath%\data'. Если же цвет не был указан ставится цвет по умолчанию, например в Geomview --- серый (R,G,B,A=0.666).

(???) Exception: each face's vertex indices are followed by an integer indicating how many color components accompany it. Face color components must be floats, not integer values. Thus a colorless triangular face might be represented as 0. Я пока не нашел таких примеров.

Есть упоминания о наличии бинарного формата, у которого заголово дополняется словом BINARY. Но примеров и описания найти пока не получилось.

Bug. В документации не получилось найти информацию о том какое поведение должно быть если местами цвет указан индексом (материал), а местами цветом с 3 или 4 компонентами.

\clearpage

\section{Формат MESH}
\section{Формат OOGL}
\section{Формат VTK}
\section{Формат NEU}
\section{Формат DAT/TEC}
\section{Формат IGES}

\section{Список форматов}
\begin{itemize}
\item OFF
\item MESH
\item OBJ
\item VTK
\item NEU
\item DAT/TEC
\item IGES
\item Прочие:
\begin{itemize}
\item QUAD
\item BBOX
\item BBP
\item BEZ
\item VECT
\item SKEL
\item LIST
\item TLIST
\item GROUP
\item DISCGRP
\item COMMENT
\end{itemize}
\item PLY
\item BYU
\item X3D
\item JVX
\item VRML
\item Maya
\item POV
\item LWO
\item 3DS
\item RIB
\item DXF
\item STL
\item ZPR
\item HDF
\item HDF5
\item NASACDF
\item NetCDF
\item FITS
\item SP3
\item DICOM
\item Affymetrix
\item BDF
\item EDF
\item MOL
\item SDF
\item SMILES
\item PDB
\item GenBank
\item FASTA
\item NDK
\item GRIB

\end{itemize}


\section{Список программ}

\begin{itemize}
\item OnReader
\item Paraview
\item Ansys Fluent
\item Ansys Icem
\item 3D max
\item Grapher
\item Tecplot
\item msh
\item CGNS
\end{itemize}

\end{document}